% pour générer un pdf: faire
% pdflatex exemple.tex
\documentclass{article}

%% Paquets LateX utiles

\usepackage[utf8]{inputenc} 		% encodage des caractères utilisés (pour les caractères accentués) 
% pour les accents, on peut soit préciser l'encodage et utiliser des caractères accentués, soit utiliser é pour un e accent aigu, \`e pour un e accent grave, etc...
%\usepackage[latin1]{inputenc} 		% autre encodage
\usepackage[french]{babel}		% pour une mise en forme "francaise"
\usepackage{amsmath,amssymb,amsthm}	% pour les maths
\usepackage{graphicx}			% pour inclure des graphiques
\usepackage{hyperref}			% si vous souhaitez que les references soient des hyperliens
\usepackage{color}			% pour ajouter des couleurs dans vos textes


% Définitions de macro
\def \R {\mathbb R}					% definit un nouveau mot cle LateX. Ici, \R designera l'ensemble des réels
\newcommand \fonctionsContinues[2] {C^0(#1,#2)}		% Une nouvelle commande avec des arguments (ici 2)


\title{Exemple LateX}
\author{Flore Nabet\footnote{\'Ecole polytechnique}}	

\begin{document}
\maketitle						% Génère le titre

\tableofcontents					% si on veut une table des matieres
\listoffigures						% si on veut la liste des figures
%\listoftables						% si on veut la liste des tableaux


\begin{center}
\textbf{Résumé}
\end{center}
Un exemple de fichier {\LaTeX} de quelques lignes.

\section{Introduction}
{\LaTeX} permet d'écrire des articles scientifiques (ou non).

\section{Les formules}

\subsection{Intégrale}

\subsubsection{En ligne}
Une formule en ligne telle que $\int_0^1 x\,dx=1/2$.

\subsubsection{Numérotée}
\begin{equation}
\label{Formule1}
\int_0^1 x\,dx=\frac 1 2.
\end{equation}

\subsubsection{Non numérotée}
\begin{equation*}
\int_0^1 x\,dx=\frac 1 2.
\end{equation*}


\subsubsection{Alignement des équations}
\begin{align}
\int_0^1 x\,dx&=\frac 1 2 \\
\int_0^1 x^2 \,dx &= \frac 1 3
\end{align}


\paragraph{Référence} 
On peut faire référence \`a la formule~\eqref{Formule1} [il faut compiler deux fois].

\subsection{Les nouvelles commandes}

\paragraph{Def} On note $\R$ l'ensemble des réels.
\paragraph{Newcommand} On désigne par $\fonctionsContinues{[0,1]}{\R}$ l'ensemble des fonctions continues de $[0,1]$ vers $\R$.

\section{Figures}
On peut aussi inclure des figures sous différents formats. La figure \ref{fig} est de format \textbf{png}.
%\begin{figure}
%\begin{center}
%\includegraphics[width=0.3\textwidth]{Decomp_spin_psi_t0_2.png}
%\includegraphics[width=1cm]{Decomp_spin_psi_t0_2.png}
%\caption{Un exemple de figures}\label{fig}
%\end{center}
%\end{figure}
{\LaTeX} optimise le placement des figures. C'est en général une mauvaise idée d'essayer de le contrarier.

\section{Conclusion}

Il faut toujours citer ses sources. Ce document a été écrit à partir d'un document initialement rédigé par Olivier Pantz \url{https://math.unice.fr/~pantz/Documents/Enseignement/LaTeX/exemple.tex}

Vous pouvez également utiliser des environnements collaboratifs (e.g. overleaf \url{https://fr.overleaf.com/}) 

On peut également réaliser des présentations de types PowerPoint avec {\LaTeX} en utilisant beamer.


\end{document}


