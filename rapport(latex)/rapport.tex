% pour générer un pdf: faire
% pdflatex exemple.tex
\documentclass{article}

%% Paquets LateX utiles

\usepackage[utf8]{inputenc} 		% encodage des caractères utilisés (pour les caractères accentués) 
% pour les accents, on peut soit préciser l'encodage et utiliser des caractères accentués, soit utiliser é pour un e accent aigu, \`e pour un e accent grave, etc...
%\usepackage[latin1]{inputenc} 		% autre encodage
\usepackage[french]{babel}		% pour une mise en forme "francaise"
\usepackage{amsmath,amssymb,amsthm}	% pour les maths
\usepackage{graphicx}			% pour inclure des graphiques
\usepackage{hyperref}			% si vous souhaitez que les references soient des hyperliens
\usepackage{color}			% pour ajouter des couleurs dans vos textes


% Définitions de macro
\def \dd {\mathrm{d}}

\title{Barrages et Risques d’Inondation}
\author{Zian Chen \& Ruikai Chen}	

\begin{document}
\maketitle						% Génère le titre

%\tableofcontents					% si on veut une table des matieres
%\listoffigures						% si on veut la liste des figures
%\listoftables						% si on veut la liste des tableaux

\section{Etude d’un seul barrage}

On a d'abord $(N_t)$ un processus de Poisson, représenté par 
\[N_t = \begin{cases}0&\textrm{pour $t\in [0, T_1[$,}\\
  1&\textrm{pour $t\in [T_1, T_2[$,}\\
  2&\textrm{pour $t\in [T_2, T_3[$,}\\
  \vdots &\quad\vdots\end{cases}\]
  
La solution est donnée par
\[X_t^1 = \exp(-r_1 t)\int_0^t \exp(-r_1 s)\dd A_s^1 + x_0^1\exp(-r_1t),\]
où $A_s^1$ est le volume cumulé au temps $s$. Puisque c'est un processus de Poisson composé, on considère l'intégrale comme une sommation. On obtient alors
\[X_t^1 = \begin{cases}x_0^1\exp(-r_1 t)&\textrm{pour $t\in [0, T_1[$,}\\
  \exp(-r_1 t)(x_0^1+\exp(r_1 T_1)U_1^1)&\textrm{pour $t\in [T_1, T_2[$,}\\
  \exp(-r_1 t)(x_0^1+\exp(r_1 T_1)U_1^1+\exp(r_1 T_2)U_2^1)&\textrm{pour $t\in [T_2, T_3[$,}\\
  \vdots &\quad\vdots\end{cases}\]
\end{document}


